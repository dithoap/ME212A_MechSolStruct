\documentclass[letter,12pt]{article}

\usepackage{subfig}

\usepackage[english]{babel}

\usepackage{fancyhdr}


\usepackage{graphicx}
\usepackage{amsmath}
\usepackage{amssymb}
\usepackage[latin1]{inputenc}
\usepackage{fancybox}
\usepackage{amsfonts}
\usepackage{bbold}
\usepackage{textcomp}


\setlength{\oddsidemargin}{0in}
\setlength{\evensidemargin}{0in}
\setlength{\textwidth}{16.1 cm}
\setlength{\topmargin}{-15mm}
\setlength{\textheight}{22cm}
\setlength{\parindent}{0cm}


\newcommand{\sv}{\underline}
\newcommand{\Sig}{\boldsymbol{\sigma}}
\newcommand{\K}{\boldsymbol{K}}
\newcommand{\Eps}{\boldsymbol{\varepsilon}}
\newcommand{\intener}{\mathcal{E}}


% --------------------------------------------------------------------------------------
% - definition des notations
% --------------------------------------------------------------------------------------


% - tenseur d'ordre 4
\newcommand{\TTTT}[1]{\underline{\underline{\underline{\underline{#1}}}}}
% - tenseur d'ordre 2
\newcommand{\TT}[1]{\underline{\underline{#1}}}
% - tenseur d'ordre 1
\newcommand{\T}[1]{\underline{#1}}

% - composante i - j de la contrainte: 
\newcommand{\sigi}[1]{\sigma_{#1}}
% - composante i - j de la deformation: 
\newcommand{\epsi}[1]{\varepsilon_{#1}}

% - tenseur de contrainte : 
\newcommand{\sig}{\TT{\sigma}}
\newcommand{\eps}{\TT{\varepsilon}}
\newcommand{\C}{\TTTT{C}}
\renewcommand{\S}{\TTTT{S}}

% - composantes des tenseurs de Hooke: 
\newcommand{\Ci}[1]{C_{#1}}
\newcommand{\Si}[1]{S_{#1}}

% - notations pour un vecteur
\renewcommand{\Vec}[1]{\underline{#1}}


% - notations pour un vecteur chapeau
\newcommand{\Vecc}[1]{\hat{\underline{#1}}}

% - notations pour un produit tensoriel
\newcommand{\tens}[2]{#1 \otimes #2}

% - notations pour les vecteurs de base
\newcommand{\e}[1]{\Vec{e_{#1}}}

% - contrainte equivalente pour les crit�res de plasticite
\newcommand{\se}{\sigma_{eq}}

% - deformation elastique
\newcommand{\epse}{\TT{\varepsilon_e}}

% - deformation elastique
\newcommand{\epsp}{\TT{\varepsilon_p}}


\begin{document}
\pagestyle{fancy}

\title{ME 211A: Infinitesimal deformation/transformation}

\maketitle

\vspace{-1cm}


\section*{Quadratic transformation}

In a plane $\left( \underline{O},{{\underline{e}}_{1}}, {{\underline{e}}_{2}} \right)$, we consider an homogeneous transformation $\underline{x}=\underline{\Phi }\left( \underline{X},t \right)$ made of the composition of:

a pure expansion about the axis $\left(\underline{O}, {\underline{e}}_{1}\right)$ with a dilatation factor $\lambda$,

a rotation of angle $\theta$ about the third axis $\left(\underline{O}, {\underline{e}}_{3}\right)$.

\begin{description}
\item \textbf{Question:} Calculate the gradient of the transformation?
\item \textbf{Answer:} $\underline{\Phi }\left( \underline{X},t \right)$ is defined by:
	\begin{equation}
	\begin{aligned}
	& {{x}_{1}}=\lambda {{X}_{1}}\cos \theta +{{X}_{2}}\sin \theta  \\ 
	& {{x}_{2}}=-\lambda {{X}_{1}}\sin\theta +{{X}_{2}}\cos \theta  \\ 
	& {{x}_{3}}={{X}_{3}} \\ 
	\end{aligned}
	\end{equation}
So the gradient of transformation is:
	\begin{equation}
	\underline{\underline{F}}=\left[ \begin{matrix}
	{{F}_{11}} & {{F}_{12}} & {{F}_{13}}  \\
	{{F}_{21}} & {{F}_{22}} & {{F}_{23}}  \\
	{{F}_{31}} & {{F}_{32}} & {{F}_{33}}  \\
	\end{matrix} \right]=\frac{\partial \underline{x}}{\partial \underline{X}}=\left[ \begin{matrix}
	\frac{\partial {{x}_{1}}}{\partial {{X}_{1}}} & \frac{\partial {{x}_{1}}}{\partial {{X}_{2}}} & \frac{\partial {{x}_{1}}}{\partial {{X}_{3}}}  \\
	\frac{\partial {{x}_{2}}}{\partial {{X}_{1}}} & \frac{\partial {{x}_{2}}}{\partial {{X}_{2}}} & \frac{\partial {{x}_{2}}}{\partial {{X}_{3}}}  \\
	\frac{\partial {{x}_{3}}}{\partial {{X}_{1}}} & \frac{\partial {{x}_{3}}}{\partial {{X}_{2}}} & \frac{\partial {{x}_{3}}}{\partial {{X}_{3}}}  \\
	\end{matrix} \right]=\left[ \begin{matrix}
	\lambda \cos \theta  & \sin \theta  & 0  \\
	-\lambda \sin \theta  & \cos \theta  & 0  \\
	0 & 0 & 1  \\
	\end{matrix} \right]
	\end{equation}
\end{description}

\begin{description}
\item \textbf{Question:} Calculate $\underline{\underline{C}}$.
\item \textbf{Answer:} 
	\begin{equation}
	\underline{\underline{C}}={{\underline{\underline{F}}}^{T}}\underline{\underline{F}}=\left[ \begin{matrix}
	{{\lambda }^{2}}  & 0  & 0  \\
	0  & 1  & 0  \\
	0 & 0 & 1  \\
	\end{matrix} \right]
	\end{equation}
\end{description}

\begin{description}
\item \textbf{Question:} Calculate $\underline{\underline{e}}$.
\item \textbf{Answer:} 
	\begin{equation}
	\underline{\underline{e}}=\frac{1}{2}\left( \underline{\underline{C}}-\underline{\underline{1}} \right)=\frac{1}{2}\left[ \begin{matrix}
	\frac{{{\lambda }^{2}}-1}{2}  & 0  & 0  \\
	0  & 0  & 0  \\
	0 & 0 & 0  \\
	\end{matrix} \right]
	\end{equation}
\end{description}

\begin{description}
\item \textbf{Question:} Calculate the displacement $\underline{\xi}$ and its gradient.
\item \textbf{Answer:} 
	\begin{equation}
	\underline{\xi }=\underline{x}-\underline{X}=\left[ \begin{matrix}
	\left( \lambda \cos \theta -1 \right){{X}_{1}}+{{X}_{2}}\sin \theta   \\
	-\lambda {{X}_{1}}\sin\theta +{{X}_{2}}\left( \cos \theta -1 \right)  \\
	0  \\
	\end{matrix} \right]
	\end{equation}
	\begin{equation}
	\underline{\underline{\nabla \xi }}=\left[ \begin{matrix}
	\lambda \cos \theta -1 & \sin \theta  & 0  \\
	-\lambda \sin\theta  & \cos \theta -1 & 0  \\
	0 & 0 & 0  \\
	\end{matrix} \right]
	\end{equation}
\end{description}

\begin{description}
\item \textbf{Question:} What is the condition for having small deformation $\left\| \underline{\underline{e}} \right\|\ll 1$? What is the physical interpretation for this condition?
\item \textbf{Answer:} 
	\begin{equation}
	\begin{aligned}
	& \left\| \underline{\underline{e}} \right\|=\sqrt{\underline{\underline{e}}:\underline{\underline{e}}} =\left|\frac{\lambda^2-1}{2}\right|\ll 1 \to |\lambda| \approx 1 \\ 
	\end{aligned}	
	\end{equation}
Thus, $\lambda \approx 1$ is the condition for having small deformation. The physical interpretation for this condition is that the expansion about axis $\left(\underline{O}, {\underline{e}}_{1}\right)$ has to be around $1$.
\end{description}

\begin{description}
\item \textbf{Question:} Explain what happens in the specific configuration $\lambda=1,\theta=\pi/2?$
\item \textbf{Answer:} 

$\underline{\underline{e}}=\underline{\underline{0}}\to$
It's rigid body rotation. It rotates $\pi/2$.
$\underline{\underline{\nabla \xi }}=\left[ \begin{matrix}
-1 & 1  & 0  \\
-1 & -1 & 0  \\
0 & 0 & 0 \end{matrix} \right]$

$\underline{\underline{\varepsilon}}=\left[ \begin{matrix}
-1 & 0  & 0  \\
0 & -1 & 0  \\
0 & 0 & 0 \end{matrix} \right]\ne\underline{\underline{e}}$, it is not small perturbation, even if it is small deformation.


\end{description}


\begin{figure}[h!]
\begin{center}

\end{center}
%\caption{The considered continuum $\Omega$}
\label{continuum}
\end{figure}

\end{document}