\documentclass[letter,12pt]{article}

\usepackage{subfig}

\usepackage[english]{babel}

\usepackage{fancyhdr}


\usepackage{graphicx}
\usepackage{amsmath}
\usepackage{amssymb}
\usepackage[latin1]{inputenc}
\usepackage{fancybox}
\usepackage{amsfonts}
\usepackage{bbold}
\usepackage{textcomp}


\setlength{\oddsidemargin}{0in}
\setlength{\evensidemargin}{0in}
\setlength{\textwidth}{16.1 cm}
\setlength{\topmargin}{-15mm}
\setlength{\textheight}{22cm}
\setlength{\parindent}{0cm}


\newcommand{\sv}{\underline}
\newcommand{\Sig}{\boldsymbol{\sigma}}
\newcommand{\K}{\boldsymbol{K}}
\newcommand{\Eps}{\boldsymbol{\varepsilon}}
\newcommand{\intener}{\mathcal{E}}


% --------------------------------------------------------------------------------------
% - definition des notations
% --------------------------------------------------------------------------------------


% - tenseur d'ordre 4
\newcommand{\TTTT}[1]{\underline{\underline{\underline{\underline{#1}}}}}
% - tenseur d'ordre 2
\newcommand{\TT}[1]{\underline{\underline{#1}}}
% - tenseur d'ordre 1
\newcommand{\T}[1]{\underline{#1}}

% - composante i - j de la contrainte: 
\newcommand{\sigi}[1]{\sigma_{#1}}
% - composante i - j de la deformation: 
\newcommand{\epsi}[1]{\varepsilon_{#1}}

% - tenseur de contrainte : 
\newcommand{\sig}{\TT{\sigma}}
\newcommand{\eps}{\TT{\varepsilon}}
\newcommand{\C}{\TTTT{C}}
\renewcommand{\S}{\TTTT{S}}

% - composantes des tenseurs de Hooke: 
\newcommand{\Ci}[1]{C_{#1}}
\newcommand{\Si}[1]{S_{#1}}

% - notations pour un vecteur
\renewcommand{\Vec}[1]{\underline{#1}}


% - notations pour un vecteur chapeau
\newcommand{\Vecc}[1]{\hat{\underline{#1}}}

% - notations pour un produit tensoriel
\newcommand{\tens}[2]{#1 \otimes #2}

% - notations pour les vecteurs de base
\newcommand{\e}[1]{\Vec{e_{#1}}}

% - contrainte equivalente pour les crit�res de plasticite
\newcommand{\se}{\sigma_{eq}}

% - deformation elastique
\newcommand{\epse}{\TT{\varepsilon_e}}

% - deformation elastique
\newcommand{\epsp}{\TT{\varepsilon_p}}


\begin{document}
\pagestyle{fancy}

\title{ME 211A: Stretch}

\maketitle

\vspace{-1cm}


\section*{3.70 Question}

Given the following right Cauchy-Green deformation tensor at a point

\begin{equation}
\left[ C \right]=\left[ \begin{matrix}
9 & 0 & 0  \\
0 & 4 & 0  \\
0 & 0 & 0.36  \\
\end{matrix} \right]
\end{equation}

\begin{description}
	
\item \textbf{(a)} Find the stretch for the material elements that were in the direction of $\textbf{e}_1$, $\textbf{e}_2$ and $\textbf{e}_3$.
\item \textbf{(b)} Find the stretch for the material element that was in the direction of $\textbf{e}_1 + \textbf{e}_2$.
\item \textbf{(c)} Find $\cos\theta$, where $\theta$ is the angle between $d\textbf{x}^{(1)}$ and $d\textbf{x}^{(2)}$ and where $d\textbf{X}^{(1)}=dS_1\textbf{e}_1$ and $d\textbf{X}^{(2)}=dS_2\textbf{e}_1$ deform into $d\textbf{x}^{(1)}=ds_1\textbf{m}$ and $d\textbf{x}^{(2)}=ds_2\textbf{n}$.

\end{description}

\begin{description}
	
\item[Solution]	
	
\item \textbf{(a)} We remind that
 \begin{equation}
	\frac{\left| \underline{v} \right|}{\left| \underline{V} \right|}=
	\frac{\sqrt{\underline{v} \cdot \underline{v}}}{\sqrt{\underline{V} \cdot \underline{V}}} = \frac{\underline{V} \cdot \underline{\underline{C}} \cdot \underline{V}}{\underline{V} \cdot \underline{V}}=\sqrt{{{C}_{ii}}}
	\label{eq:first}
 \end{equation} 

Thus, stretch for direction $\underline{e}_{1}$ is $\sqrt{{C}_{11}}=3$, stretch for direction $\underline{e}_{2}$ is $\sqrt{{C}_{22}}=2$, stretch for direction $\underline{e}_{3}$ is $\sqrt{{C}_{33}}=0.6$,

\item \textbf{(b)}

To get stretch for direction $\underline{e}_{1}+\underline{e}_{2}$, we define $|\underline{V}| = \frac{1}{\sqrt{2}}(\underline{e}_{1}+\underline{e}_{2})$, so we have
\begin{equation}
\underline{v} \cdot \underline{v} = \underline{V} \cdot \underline{\underline{C}} \cdot \underline{V} = \frac{1}{2} \cdot \left( \begin{matrix}1\\1\\0\\\end{matrix} \right) \left[ \begin{matrix}9 & 0 & 0\\0 & 4 & 0\\0 & 0 & 0.36\\\end{matrix} \right] \left( \begin{matrix}1\\1\\0\\\end{matrix} \right) = \frac{13}{2}
\end{equation}
So the stretch is $\frac{13}{2}$.

\item \textbf{(c)}

\begin{equation}
	\sin \theta = \frac{|\underline{dx}^{(1)}| \cdot  |\underline{dx}^{(2)}|}{|\underline{dX}^{(1)}| \cdot  |\underline{dX}^{(2)}|} = \frac{dS_1 \underline{e}_{1} \cdot \underline{\underline{C}} \cdot \underline{e}_{2}dS_2}{dS_1dS_2} = C_{12} = 0
\end{equation}

Thus no angle variation, $\cos \theta = 1$.

\end{description}




\end{document}